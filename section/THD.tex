% !TeX spellcheck = de_DE

\section{Thermodynamik}
%
	\setlength{\abovedisplayskip}{-15pt}
		\[ \arraycolsep=1em % \def\arraystretch{2}
		\begin{array}{llll}
			\text{isotherm:}  & \Delta u=0           & \Delta T=0                               & \Delta w = + R\ T  \ln \left(\dfrac{v_1}{v_2} \right) = - R\ T \ln \left(\dfrac{p_1}{p_2}\right) \\
			\text{isobar:}    & \Delta w=R\ \Delta T & \Delta p=0 \Rightarrow \int{}v\dd p=0    & \Delta h = \Delta q_a \text{ wenn } \Delta q_R=0                                                 \\
			\text{isochor:}   &                      & \Delta v=0 \Rightarrow \int p\dd v=0     & \Delta u = \Delta q_a \text{ wenn } \Delta q_R=0                                                 \\
			\text{Enthalpie:} & h=u+pv               & %\multicolumn{1}{r}{\text{Molare Masse:}} & M = \dfrac{m}{n}
		\end{array} \]
	%
	\setlength{\abovedisplayshortskip}{-10pt}
		\[ \arraycolsep=1em  \def\arraystretch{1.4}
		\begin{array}{llll}
			\textbf{1. HS:} & \Delta h = \Delta q_a + \Delta q_R + \int v\ \dd p    & \Delta u = \Delta q_a + \Delta q_R - \int p\ \dd v & \Delta Q + \Delta W = \Delta U + \Delta E_a \\
			\textbf{2. HS:} & \Delta q_{rev}=\Delta q_a + \Delta q_R= \int T\ \dd s &                                                    &
		\end{array} \]

\subsection{Ideales Gas}
%
	\setlength{\abovedisplayskip}{-20pt}
		\[ \arraycolsep=1em  \def\arraystretch{1.5}
		\begin{array}{llll}
			v_{mn}=22,414\, \dfrac{\si{Nm^3}}{\si{\kmol}}          & V_n = n\ v_{mn}                                                                               & R=\dfrac{\mathrm{R_m}}{M}       & \kappa = \dfrac{c_p}{c_v} \\
			\mathrm{R_m} = \qty{8314,47}{\J\per\kmol\per\K}        & p\ v=R\ T                                                                                     & p\ V = m\ R\ T         & p\ V=n\ \mathrm{R_m}\ T            \\
			c_p=\dfrac{c_{pm}}{M}=R+c_v=R \dfrac{\kappa}{\kappa-1} & c_p=c_p|_{t_1}^{t_2}  =   \dfrac{c_p|_{0}^{t_2}\cdot t_2 - c_p|_{0}^{t_1}\cdot t_1}{t_2- t_1} & \Delta u=c_v\ \Delta T & \Delta h=c_p\ \Delta T
		\end{array} \]
%
	\setlength{\abovedisplayshortskip}{-14pt}
		\[ \arraycolsep=1em  \def\arraystretch{2.7}
		\begin{array}{lll}
	  \Delta s = c_v \ln\left( \dfrac{T_2}{T_1}\right) + R \ln\left( \dfrac{v_2}{v_1}\right)
			&  = c_p \ln\left( \dfrac{T_2}{T_1}\right) - R \ln\left( \dfrac{p_2}{p_1}\right)
			&  = c_p \ln\left( \dfrac{v_2}{v_1}\right) + c_v \ln\left( \dfrac{p_2}{p_1}\right)
		\end{array} \]

		\[ \arraycolsep=0.6em  \def\arraystretch{2.7}
		\begin{array}{lll}
			\text{Isentrope:}
			& \dfrac{T_{2s}}{T_1}\   =  \left(\dfrac{v_1}{v_2}\right)^{\kappa-1}  =
				\left(\dfrac{p_2}{p_1}\right)^\frac{\kappa-1}{\kappa}
			& p\ v^\kappa  =  p_1\ v_1^\kappa  =  \text{konst.}
		\\
			& \int p\ \dd v = \dfrac{1}{\kappa - 1}  R \ T_1
				\left[1\ -\ \left(\dfrac{v_1}{v_2}\right)^{\kappa-1}\right]
			& \int v\ \dd p = \dfrac{\kappa}{\kappa-1}  R \ T_1
				\left[\left(\dfrac{p_2}{p_1}\right)^\frac{\kappa-1}{\kappa}-1\right]
		\\
			\text{Polytrope analog mit:}
			&  \text{n} = \dfrac{
				\ln\left(\dfrac{p_2}{p_1}\right)} {
				\ln\left(\dfrac{p_2}{p_1}\right) -
				\ln\left(\dfrac{T_2}{T_1}\right)}
			&  = 1 - \dfrac{
				\ln\left(\dfrac{T_2}{T_1}\right)} {
				\ln\left(\dfrac{v_1}{v_2}\right)} \text{ statt } \kappa
		\end{array} \]

\subsection{Gemische idealer Gase -- Species \textit{i}}
	\setlength{\abovedisplayshortskip}{-15pt}
		\[ \arraycolsep=1em % \def\arraystretch{2}
		\begin{array}{llll}
			y_i = \dfrac{n_i}{n} = \dfrac{V_i}{V} = \dfrac{\dot{V_i}}{\dot{V}} = \dfrac{p_i}{p} & w_i  =  \dfrac{m_i}{m}  = \dfrac{\dot{m}_i}{\dot{m}} =  y_i \dfrac{M_i}{M} & M = \sum y_i\ M_i          & M_i = \dfrac{m_i}{n_i} \\
			c_p = \sum c_{pi}\ w_i                                                              & \text{analog für: }  c_v,\ \Delta u,\ \Delta h,\ \Delta s                  & c_{mp} = \sum y_i\ c_{mpi} & \dot{V} = A\ c
		\end{array} \]

\subsection{Inkompressible Flüssigkeiten}
	\setlength{\abovedisplayshortskip}{-20pt}
		\[ \arraycolsep=2em % \def\arraystretch{2}
		\begin{array}{lll}
			c_v = c_p               & v = \dfrac{1}{\varrho} = \text{konst.} & \dot{m} = \dot{V}\ \varrho                      \\
			\Delta u = c_p \Delta T & \Delta h = c_p \Delta T + v \Delta p   & \Delta s = c_p \ln\left(\dfrac{T_2}{T_1}\right)
		\end{array} \]

\subsection{Gemische mischbarer, inkompressibler Flüssigkeiten -- Species \textit{i}}
	\setlength{\abovedisplayskip}{-15pt}
		\[  \arraycolsep=2em  \def\arraystretch{1.5}
		\begin{array}{lll}
			m_i = \dfrac{V_i}{v_i} & v   = \sum v_i w_i                                 & m   = \sum m_i   \qquad\qquad   w_i = \dfrac{m_i}{m}                                                    \\
			c_p = \sum c_{pi} w_i  & \text{analog für}   \Delta u,\ \Delta h,\ \Delta s & \Delta s_i = c_{pi} \ln\left(\dfrac{T_{2i}}{T_{1i}}\right) + R_i \ln\left(\dfrac{v_{2i}}{v_{1i}}\right) \\
			x_i = w_i \dfrac{M}{M_i} = \varphi_i \dfrac{v_i\ M_i}{v\ M} & w_i = x_i \dfrac{M_i}{M} = \varphi_i \dfrac{v}{v_i} & \varphi_i = x_i \dfrac{v\ M}{v_i\ M_i} = w_i \dfrac{v_i}{v}
		\end{array} \]


\subsection{Nassdampf:  \textit{u}'  = Wasser,  \textit{u}''  = Dampf}
%
	\setlength{\abovedisplayshortskip}{-18pt}
		\[ \arraycolsep=2em  \def\arraystretch{1.5}
		\begin{array}{lll}
			                      & x = \dfrac{m''}{m'+m''} = \dfrac{u-u'}{u''-u'} &                                \\
			u = (1-x)\ u' + x u'' & u = u' + x\ (u''-u')                           & \text{analog für: }  v,\ h,\ s
		\end{array} \]
%
\vspace {-1em}
%
\begin{flushleft}
	\setlength{\tabcolsep}{0em} % for the horizontal padding
	\begin{tabular}{lp{4em}l}
		\parbox{6cm}{\subsection{Geschlossene Systeme}} &  & \parbox{5cm}{\subsection{Offene Systeme}}                                                    \\
		\qquad  $ \Delta w_v = -\int p \ \dd v $        &  & \qquad $ \Delta U + \Delta E_a = \Delta Q_a + \Delta W_i + \sum \Delta m_j \ (h_j+ e_{aj}) $
	\end{tabular}
\end{flushleft}

\subsection{Einseitig offene Systeme}
%
	\setlength{\abovedisplayskip}{-20pt} % todo WARUM IST HIER MEHR INDENT #@$%^*&$#%@%!!! Wegen flushleft oben?
		\[ \arraycolsep=2em % \def\arraystretch{2}
		\begin{array}{lll}
			\text{Einströmung -- ideales Gas:} &  T_2 = T_1\ \dfrac{\kappa}{1+\dfrac{p_1}{p_u}(\kappa-1)}  &  \Delta m = \dfrac{m_1\ (T_2-\ T_1)}{\kappa\ T_1-\ T_2}  \\
			\text{Einsaugen, Ausstoßen:}       &  \Delta W_v = -p\ \Delta V                                &
		\end{array} \]

\subsection{Ruhende, stationäre, 2-seitig offene Systeme}
%
	\setlength{\abovedisplayskip}{-20pt}
	\[\arraycolsep=2em
		\begin{array}{lll}
			\Delta h + \Delta e_a = \Delta q_a + \Delta w_i & \Delta h = \Delta q_a + \Delta q_R + \int v\ \dd p & \int v\ \dd p + \Delta q_R + \Delta e_a = \Delta w_i
		\end{array} \]
\subsection{Wirkungsgrade}
%
	\setlength{\abovedisplayskip}{-25pt}
		\[ \arraycolsep=1em \def\arraystretch{2}
		\begin{array}{lllll}
			\text{Verdichtung:}      & \eta_{is} = \dfrac{\Delta w_{s=konst.}}{\Delta w_i} & \eta_{it} = \dfrac{\Delta w_{T=konst.}}{\Delta w_i} & \eta_a    = \dfrac{\Delta w_i}{\Delta w_{\eff}} & \eta_{\eff} = \eta_i \ \eta_a \\
			\text{Entspannung:}      & \eta_{is} = \dfrac{\Delta w_i}{\Delta w_{s=konst.}} & \eta_{it} = \dfrac{\Delta w_i}{\Delta w_{T=konst.}} & \eta_a    = \dfrac{\Delta w_{\eff}}{\Delta w_i} & \eta_{\eff} = \eta_i \ \eta_a \\
			\text{Wärmeübertragung:} & \eta_{wue} = \dfrac{\dot{Q}}{\dot{Q}_{max}}         &                                                     &                                                &
		\end{array} \]

\subsection{Ruhende, stationäre, 3-seitig offene Systeme}
%
	\setlength{\abovedisplayskip}{-15pt}
	\[ \arraycolsep=1em \def\arraystretch{1}
		\setlength{\tabcolsep}{1em} % for the horizontal padding
		\begin{array}{ll}
			\text{Mischung der Ströme i:} &  \sum m_i\ \Delta h_i = 0
		\end{array} \]

\subsection{Ruhende, stationäre, 4-seitig offene Systeme}
	%
	\setlength{\abovedisplayskip}{-15pt}
	\[ \arraycolsep=1em \def\arraystretch{1.2}
		\begin{array}{lll}
			\text{Wärmeübertrager:}      & \sum m_i\ \Delta h_i = 0 &                                 \\
			\text{Verdichter + Turbine:} & P_{vi} + P_{T,\eff} = 0  & \text{montiert auf einer Welle}
		\end{array} \]

\subsection{Kreisprozesse}
%
	\setlength{\abovedisplayshortskip}{-25pt}
	\[ \arraycolsep=1em \def\arraystretch{1.8}
		\begin{array}{llll}
			\text{Allgemein:} & \eta_{therm} = \dfrac{\abs{\Delta w_{ab}}} {\Delta q_{zu}} & \text{EER} = \dfrac{\Delta q_{zu}}{\Delta w_{zu}}    & \text{COP} = \dfrac{\abs{\Delta w_{ab}}}{\Delta w_{zu}} \\
			\text{Carnot:}    & \eta_{therm} = 1- \dfrac{T_{ab}}{T_{zu}}                   & \text{EER} = \dfrac{T_{zu}}{T_{ab} - T_{zu}}         & \text{COP} = \dfrac{T_{ab}}{T_{ab} - T_{zu}}            \\
			\text{Joule:}     & \eta_{therm,max} = 1 - \dfrac{T_1}{T_2}                    &                                                      &                                                         \\
			\text{Gütegrad:}  & \eta_G = \dfrac{\mathrm{EER}}{\mathrm{EER}_{Carnot}}       & \eta_G = \dfrac{\mathrm{COP}}{\mathrm{COP}_{Carnot}} &
		\end{array} \]

\subsection{Exergie}

	\hspace{3cm}
	\setlength{\abovedisplayshortskip}{-20pt}
	\[ \arraycolsep=1em \def\arraystretch{1.3}
		\begin{array}{ll}
			\text{Arbeit:}                     & e = \Delta w_{\eff}                                                                                                       \\
			\text{Geschlossenes System:}       & e = u - u_u - T_u\ (s-s_u) - p_u\ (v_u-v)                                                                                \\
			\text{Fluid Strom}:                & e = h - h_u - T_u\ (s-s_u)                                                                                               \\
			\text{Wärme:}                      & e = \mathop{\mathlarger{\int}} \left(1-\dfrac{T_u}{T} \right) \dd q_a  \cong \left(1-\dfrac{T_u}{T_m} \right) \Delta q_a \\
			\text{Exergetischer Wirkungsgrad:} & \zeta = \dfrac{e_{ab}}{e_{zu}}
		\end{array} \]
